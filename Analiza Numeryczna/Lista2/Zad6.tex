\documentclass{article}

\title{Lista 2 Zadanie 6}
\author{Dominik Budzki, nr. indeksu 314625}

\begin{document}
    \maketitle

Załózmy, ze x, y sa liczbami maszynowymi. Podaj przykład pokazujacy, ze przy obliczaniu wartosci
d:= $\sqrt{x^2 + y^2}$
algorytmem postaci\\
u:= $x*x$\\
u:=$u+y*y$\\
d:=sqrt(u)\\
moze wystapic zjawisko nadmiaru, mimo tego, ze szukana wielkosc d nalezy do zbioru
$X_{fl}$. Nastepnie zaproponuj algorytm wyznaczania d pozwalajacy unikac zjawiska nadmiaru,
jesli $\sqrt{2}$ max($|x|, |y|$) $\in X_{fl}$. Na koniec podaj skuteczna metode wyznaczania
długosci euklidesowej wektora $v \in R^n$.\\

Rozwiazanie:\\
Zalozmy ze najwieksza wartosc $X_{fl}$ to $2^{32}$ i wezmy takie x i y, że:\\
$x = y = 2^{30}$\\
$u = 2^{60}$\\
$u = 2^{60} + 2^{60}$(*)\\
$d = 2^{30}\sqrt{2}$\\
\\
Przy rownaniu oznaczonym gwiazdka widac ze wykonujemy obliczenia na liczbie spoza $X_{fl}$, wiec wystepuje nadmiar mimo ze $2^{30}\sqrt{2} \in X_{fl}$
\\

Mozna przeksztalcic rownanie d w taki sposob:\\
$d = \sqrt{x^2 + y^2} = \sqrt{x^2(1+\frac{y^2}{x^2})} = |x|\sqrt{1+\frac{y^2}{x^2}}$\\
gdy $x \geq y$ to $\sqrt{1+\frac{y^2}{x^2}} \leq \sqrt{2}$\\
x i y mozna traktowac zamiennie, wylaczamy wieksza z nich. Wtedy mamy $\sqrt{2}$ max($|x|, |y|$) $\in X_{fl}$ i nie wystepuje zjawisko nadmiaru.\\

Wyznaczanie dlugosci euklidesowej wektora $v \in R^n$.\\
$v = (x_1, x_2, ..., x_n)$\\
$||v|| = \sqrt{\Sigma_{i=1}^n x_i^2}$\\
teraz podobnie jak wyzej z wartoscia $d$ wylaczamy przed nawias najwieksze $x_k$
$\sqrt{x_1^2+x_2^2+...+x_n^2} = \sqrt{x_k^2(\frac{x_1^2}{x_k^2}+\frac{x_2^2}{x_k^2}+...+\frac{x_n^2}{x_k^2})} = |x_k|\sqrt{(\frac{x_1^2}{x_k^2}+\frac{x_2^2}{x_k^2}+...+\frac{x_n^2}{x_k^2})} \approx $\\
$ \approx |x_k|\sqrt{n}$\\
$ v = max(x_1, x_2, ..., x_n)\sqrt{n}$\\

\end{document}