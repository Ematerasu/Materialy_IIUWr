\documentclass{article}

\title{Lista 2 Zadanie 9}
\author{Dominik Budzki, nr. indeksu 314625}

\begin{document}
\maketitle

Mozna wykazac, ze przy $x_1 = 2$ ciag\\

\begin{center}
    $x_{k+1} = 2^k\sqrt{2(1 - \sqrt{1 - (x_k/2^k)^2})}$
\end{center}

jest zbiezny do $\pi$. Czy podczas obliczania kolejnych wyrazów tego ciagu przy pomocy
komputera moze wystapic zjawisko utraty cyfr znaczacych? Jesli tak, to zaproponuj
inny sposób wyznaczania wyrazów tego ciagu pozwalajacy uniknac wspomnianego zjawiska.
Przeprowadz odpowiednie testy obliczeniowe.\\

Rozwiazanie:\\
Dla k = 29 $x_{29} = 0$, gdyz $\frac{x_{28}}{2^{28}} \approx 0$ bo zostaja uciete znaczace cyfry.\\
Zeby zapobiec temu mozemy przeksztalcic fragment tej formuly a dokladnie $1 - \sqrt{1 - (x_k/2^k)^2}$\\

$1 - \sqrt{1 - (x_k/2^k)^2} = \frac{(1 - \sqrt{1 - (x_k/2^k)^2})(1 + \sqrt{1 - (x_k/2^k)^2})}{1 - \sqrt{1 + (x_k/2^k)^2}} = $\\
$ = \frac{1 - 1 + (x_k/2^k)^2}{1 + \sqrt{1 - (x_k/2^k)^2}} =  \frac{(x_k/2^k)^2}{1 + \sqrt{1 - (x_k/2^k)^2}} = \frac{x_k^2}{2^{2k}(1 + \sqrt{1 - (x_k/2^k)^2})}$\\

Wstawmy teraz ten fragment w formule: \\
$x_{k+1} = 2^k\sqrt{2\frac{x_k^2}{2^{2k}(1 + \sqrt{1 - (x_k/2^k)^2})}} = \sqrt{2\frac{x_k^2}{1 + \sqrt{1 - (x_k/2^k)^2}}}$\\

Teraz dziala poprawnie dla duzych k.


\end{document}